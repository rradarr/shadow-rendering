\chapter{Summary}
\label{section:chapter_5}

% \begin{itemize}
% \item What problem have I solved?
% \item How have I solved the problem?
% \item What are pros and cons of my solutions?
% \item Can I state some recommendations?
% \end{itemize}

% \begin{itemize}
% \item synthetic description of performed work
% \item conclusions
% \item  future development, potential future research
% \item Has the objective been reached?
% \end{itemize}

Within the realm of computer graphics rendering, many approaches to render various physical phenomena exist. Shadow rendering is no exception, offering a plethora of choices varying in visual robustness, achievable performance and realism. This thesis analyzed some of the techniques meant for real-time shadow rendering, providing their background and explaining their algorithms. Within the scope of the thesis, a test application was built in \textit{C++} using the \textit{DirectX 12} graphics API. The application implemented chosen techniques and allowed them to be tested and compared against each other on the basis of performance and visual quality.

The tests demonstrated that each technique has its pros and cons. The best  quality of the rendered results in percentage-closer soft shadows is reflected in the heavy performance requirements. Fast basic shadow mapping would look out of place within a modern renderer, giving unrealistic hard shadows and heavy aliasing. A solid middle ground seems to be variance shadow mapping, which is exceptionally performant and can eliminate aliasing entirely. Adaptive percentage-closer filtering produces pleasant results and attempts to improve the performance of basic PCF with a clever branching algorithm. Albeit offering worse performance than VSM, it can retain more detail in the shadows and does not suffer from light bleeding.

Within the scope of this thesis six approaches to shadow rendering were implemented and tested. In the future it would be worthwhile to investigate more advanced and modern rendering techniques to see how some problems present in the ones already covered are getting solved and in what directions the field is heading.
\chapter{Problem analysis} % The title of this chapter is similar to the title of the thesis.!

\section{The task of rendering}
% the old boring intro

In this thesis, rendering is understood as the process of obtaining an image from a description of a three-dimensional scene. A scene can be rendered in a multitude of ways, with differences both in the specifics of the initial scene description and the rendered result. The rendered visuals can range from stylized to photorealistic. Rendering happens everywhere where a computer generated image is created for a viewer to see.

The wide spectrum of possible rendering results hints at the multitude of ways in which rendering is actually performed. There are various techniques employed during the rendering process, which can be utilized together to create a desired look and fit within performance constraints. In more complex processes, there are many techniques used to render a scene, each responsible for modeling the visual aspects of different phenomena or effects.


Different current techniques for shadow rendering?

\begin{itemize}
\item What problem do I want (have to :-) to solve?
\item Why the problem is important?
\item How do others solve the problem?
\item What are pros and cons of my solution?
\end{itemize}

References to 
book \cite{bib:book},
scientific papers in journals \cite{bib:article},
conference papers \cite{bib:conference},
and web pages \cite{bib:internet}.

Equations should be numbered:
\begin{align}
y = \frac{\partial x}{\partial t}
\end{align}

\chapter{[Problem analysis]}

\begin{itemize}
\item problem analysis, problem statement
\item state of the art, literature research (all sources in the thesis have to be referenced, eg journal article \cite{bib:article} book \cite{bib:book}, conference paper \cite{bib:conference}, internet source \cite{bib:internet})
\item description of known solutions, algorithms
\item location of the thesis in scientific domain
\item The title of this chapter is similar to the title of the thesis.
\end{itemize}

\begin{Definition}\label{def:definition}
body of the definitions
\end{Definition}

\begin{Theorem}[optional name]\label{t:theorem}
body of the theorem
\end{Theorem}

\begin{Example}[optional name]\label{ex:example}
body of the example
\end{Example}

%%%%%%%%%%%%%%%%%%%%%%%%



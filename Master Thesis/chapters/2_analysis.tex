\chapter{Problem analysis} % The title of this chapter is similar to the title of the thesis.!

\section{The task of rendering}
% the old boring intro

In this thesis, rendering is understood as the process of obtaining an image from a description of a three-dimensional scene. A scene can be rendered in a multitude of ways, with differences both in the specifics of the initial scene description and the rendered result. The rendered visuals can range from stylized to photorealistic. Rendering happens everywhere where a computer generated image is created for a viewer to see.

The wide spectrum of possible rendering results hints at the multitude of ways in which rendering is actually performed. There are various techniques employed during the rendering process, which can be utilized together to create a desired look and fit within performance constraints. In more complex processes, there are many techniques used to render a scene, each responsible for modeling the visual aspects of different real-life phenomena or artificial effects. A renderer could be capable of adding stylized edge detection and cell-shading to an image, rendering glossy and rough surfaces, simulating shadows, reflections, caustics and refraction, dealing with hair and fur or volumetric participating media such as fog, smoke and clouds. Each of the mentioned effects can be rendered using one of many techniques, and many of them are being actively improved upon and researched.

The fact that there are many techniques currently in use to render a single type of effect creates and advantageous situation where the techniques used can be chosen for each application, depending on its specifics. Two main factors can be discerned as defining the needs of an application: the performance requirements and the desired visual style.

The desired visual style is partially just a matter of preference defined by the style of the project. More importantly however it is a matter of clearly and efficiently conveying visual information, in a way that is consistent with the rest of the use case and with what the end user expects. This means that realism, often touted the pinnacle and goal of computer graphics, is not necessarily always the best approach. A CAD (computer-aided design) application or a 3D modelling tool would be made less useful by including realistic reflections and highly contrasting, full shadows in the rendered viewport, as these programs need to clearly convey information about the shape and design of 3D objects, without distractions and obstructions. On the other hand, when a player starts a modern action-adventure game, they expect a level of realism in the game's graphics that allows for immersion in the presented world. The intricate visuals also make for a more engaging experience and can mean a better reception of a game. At the same time, design choices should be made to ensure that the realism or intricacy of the presented graphics does not get in the way of enjoying the gameplay, which in the end should be the main attribute of a video game.

% now perf

Different current techniques for shadow rendering?

\begin{itemize}
\item What problem do I want (have to :-) to solve?
\item Why the problem is important?
\item How do others solve the problem?
\item What are pros and cons of my solution?
\end{itemize}

References to 
book \cite{bib:book},
scientific papers in journals \cite{bib:article},
conference papers \cite{bib:conference},
and web pages \cite{bib:internet}.

Equations should be numbered:
\begin{align}
y = \frac{\partial x}{\partial t}
\end{align}

\chapter{[Problem analysis]}

\begin{itemize}
\item problem analysis, problem statement
\item state of the art, literature research (all sources in the thesis have to be referenced, eg journal article \cite{bib:article} book \cite{bib:book}, conference paper \cite{bib:conference}, internet source \cite{bib:internet})
\item description of known solutions, algorithms
\item location of the thesis in scientific domain
\item The title of this chapter is similar to the title of the thesis.
\end{itemize}

\begin{Definition}\label{def:definition}
body of the definitions
\end{Definition}

\begin{Theorem}[optional name]\label{t:theorem}
body of the theorem
\end{Theorem}

\begin{Example}[optional name]\label{ex:example}
body of the example
\end{Example}

%%%%%%%%%%%%%%%%%%%%%%%%



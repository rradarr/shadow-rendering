\chapter{Technical documentation}

\section{Building the test application}
The test application uses CMake to get all the dependencies and create a buildable MSVC project. The application was built using Visual Studio Code and CMake Tools extension. In the CMake Tools extension tab the configuration, for example `msvc Release', should be selected. After that the build target `ShadowRendering' and a preset have to be set. The extension will then begin fetching the dependencies and configuring the project. Once ready the program can be built. The only dependency that is not handled by CMake is the Windows SDK. The user has to install it manually in the default installation location. The executable is built into an appropriate location in the /build directory, depending on the configuration.

\section{Using the test application}
Once launched, the application will load the Cube scene and present a window containing the rendered output. ImGUI can be toggled with the H key. It contains information on controls, allows selection of rendering modes and provides brief explanations. It also displays the FPS and frame time of the application if the widget is enabled.

\section{Profiling the test application}
To profile the test application, an executable of TracyProfiler can be downloaded from its GitHub repository. First the tracy-profiler.exe should be run, then the test application. The test application should be listed in Tracy as ready for profiling. Upon connecting, data gathering will begin immediately and frames will start appearing in the Tracy GUI. They can be later analyzed and inspected, for example by checking how the GPU is utilized, inspecting the time the CPU is waiting for the GPU or analyzing the timing statistics that Tracy provides.
\chapter{Experiments}
% This chapter presents the experiments. It is a crucial part of the thesis and has to dominate in the thesis. 
% The experiments and their analysis should be done in the way commonly accepted in the scientific community (eg. benchmark datasets, cross validation of elaborated results, reproducibility and replicability of tests etc).

\section{Methodology}
% \begin{itemize}
% \item description of methodology of experiments
% \item description of experimental framework (description of user interface of research applications – move to an appendix)
% \end{itemize}
One of the objectives of the thesis was to compare multiple shadow rendering methods both in a qualitative and quantitative ways. To achieve this a test renderer application was created, which implements the rendering techniques and allows a user to observe their results as well as profile the performance the program.

The tests were performed in a semi-controlled environment. Care was taken to avoid other processes interrupting and affecting the test results, but their impact could not be entirely eliminated in. Results of repeated tests at different moments in time or on different machines with the same hardware will differ, but that difference should not be large enough to impact the overall comparison results. 
% TODO: if you use two machines, mention it here.
The machine used for performing measurements had the following specification:
\begin{itemize}
    \item OS: Windows 10, 22H2
    \item CPU: AMD Ryzen 5 1600
    \item GPU: Nvidia GTX 1080, 8 GB VRAM
    \item RAM: 16 GB
\end{itemize}

\subsection{Created test application}
The test application allows the user to choose a rendering mode and observe the results in real time. It also allows to change the viewpoint and observe how the shadows behave in motion. Most rendering modes implement different shadow rendering techniques and some include debug views like wireframe mode or mesh colored by normals. The application is also instrumented with profiling commands which make it possible to connect to an external profiler and collect data in real time, as well as save and load existing profiling data.

The application was written in the C++ programming language. It uses the Win32 platform to create windows in a Windows operating system. It uses DirectX 12 to utilize the GPU for rendering graphics. DirectX 12 was chosen because it is the newest of DirectX versions, allowing for granular control over the rendering process and boasting advanced features. It is also very widely adopted for all sorts of computer graphics tasks, making it worth specializing in. Additionally, the DirectX 12 API is much more verbose and involved as compared to, for example, OpenGL, which forces the programmer to gain more in-depth knowledge about graphics programming and all the processes involved.

Two notable open-source libraries were utilized in the project. Dear ImGui is a library simplifying the creation of graphical user interfaces. It is compatible with all the major graphics programming APIs, including DX12, making it fast and easy to integrate into a project, providing am intuitive and smooth user experience at the same time. The other library is the Tracy Profiler. The instrumentation library is just a part of the Tracy ecosystem, with separate applications like the profiler to gather and analyze performance data. It is simple to incorporate into an existing project and very performant, adding almost no overhead. Code zones and frames that will be captured when profiling are marked manually by the programmer, giving them full control over what data is captured. This also improves the readability of the outputs, which stands in contrast with other profilers that automatically gather data and can easily overwhelm the user and clutter the presentation with the amount of information provided.

\section{Data sets}
The data sets consist of multiple scenes used for testing.
% TODO: describe the scenes.

\section{Results}
% \begin{itemize}
% \item presentation of results, analysis and wide discussion of elaborated results, conclusions
% \end{itemize}

The results of testing each implemented shadow rendering method are presented in the following sections. Each set of results is complemented by a description and discussion, highlighting possible causes and, if appropriate, possible fixes. The techniques might undergo different tests based on their characteristic.

\subsection{Planar shadow mapping}
% TODO: do I want to even test this?

\subsection{Basic shadow maps}
% TODO: test them with different scenes (complexity and scale), different shadow map resolutions, different biases

% \begin{table}
% \centering
% \caption{A caption of a table is ABOVE it.}
% \label{id:tab:wyniki}
% \begin{tabular}{rrrrrrrr}
% \toprule
% 	         &                                     \multicolumn{7}{c}{method}                                      \\
% 	         \cmidrule{2-8}
% 	         &         &         &        \multicolumn{3}{c}{alg. 3}        & \multicolumn{2}{c}{alg. 4, $\gamma = 2$} \\
% 	         \cmidrule(r){4-6}\cmidrule(r){7-8}
% 	$\zeta$ &     alg. 1 &   alg. 2 & $\alpha= 1.5$ & $\alpha= 2$ & $\alpha= 3$ &   $\beta = 0.1$  &   $\beta = -0.1$ \\
% \midrule
% 	       0 &  8.3250 & 1.45305 &       7.5791 &    14.8517 &    20.0028 & 1.16396 &                       1.1365 \\
% 	       5 &  0.6111 & 2.27126 &       6.9952 &    13.8560 &    18.6064 & 1.18659 &                       1.1630 \\
% 	      10 & 11.6126 & 2.69218 &       6.2520 &    12.5202 &    16.8278 & 1.23180 &                       1.2045 \\
% 	      15 &  0.5665 & 2.95046 &       5.7753 &    11.4588 &    15.4837 & 1.25131 &                       1.2614 \\
% 	      20 & 15.8728 & 3.07225 &       5.3071 &    10.3935 &    13.8738 & 1.25307 &                       1.2217 \\
% 	      25 &  0.9791 & 3.19034 &       5.4575 &     9.9533 &    13.0721 & 1.27104 &                       1.2640 \\
% 	      30 &  2.0228 & 3.27474 &       5.7461 &     9.7164 &    12.2637 & 1.33404 &                       1.3209 \\
% 	      35 & 13.4210 & 3.36086 &       6.6735 &    10.0442 &    12.0270 & 1.35385 &                       1.3059 \\
% 	      40 & 13.2226 & 3.36420 &       7.7248 &    10.4495 &    12.0379 & 1.34919 &                       1.2768 \\
% 	      45 & 12.8445 & 3.47436 &       8.5539 &    10.8552 &    12.2773 & 1.42303 &                       1.4362 \\
% 	      50 & 12.9245 & 3.58228 &       9.2702 &    11.2183 &    12.3990 & 1.40922 &                       1.3724 \\
% \bottomrule
% \end{tabular}
% \end{table}  

% The table is here too \ref{id:tab:wyniki}

%%%%%%%%%%%%%%%%%%%%%
% FIGURE FROM FILE
%
% \begin{figure}
% \centering
% \includegraphics[width=0.5\textwidth]{./graf/politechnika_sl_logo_bw_pion_en.pdf}
% \caption{Caption of a figure is always below the figure.}
% \label{fig:label}
% \end{figure}

% Fig. \ref{fig:label} presents asdasd

% \begin{figure}
% \includegraphics[width=0.5\textwidth]{./graf/test_image.jpg}
% \caption{Caption of a figure is always below the figure akakak.}
% \label{fig:duped_image3}
% \end{figure}

% some citation of the above \ref{fig:duped_image3}

%%%%%%%%%%%%%%%%%%%%%
%
%%%%%%%%%%%%%%%%%%%%
%% SUBFIGURES
%
% \begin{figure}
% \centering
% \begin{subfigure}{0.4\textwidth}
%    \includegraphics[width=\textwidth]{./graf/politechnika_sl_logo_bw_pion_en.pdf}
%    \caption{Upper left figure.}
%    \label{fig:upper-left}
% \end{subfigure}
% \hfill
% \begin{subfigure}{0.4\textwidth}
%    \includegraphics[width=\textwidth]{./graf/politechnika_sl_logo_bw_pion_en.pdf}
%    \caption{Upper right figure.}
%    \label{fig:upper-right}
% \end{subfigure}

% \begin{subfigure}{0.4\textwidth}
%    \includegraphics[width=\textwidth]{./graf/politechnika_sl_logo_bw_pion_en.pdf}
%    \caption{Lower left figure.}
%    \label{fig:lower-left}
% \end{subfigure}
% \hfill
% \begin{subfigure}{0.4\textwidth}
%    \includegraphics[width=\textwidth]{./graf/politechnika_sl_logo_bw_pion_en.pdf}
%    \caption{Lower right figure.}
%    \label{fig:lower-right}
% \end{subfigure}
       
% \caption{Common caption for all subfigures.}
% \label{fig:subfigures}
% \end{figure}
% Fig. \ref{fig:subfigures} presents very important information, eg. Fig. \ref{fig:upper-right} is an upper right subfigure.
%%%%%%%%%%%%%%%%%%%%%



% \begin{figure}
% \centering
% \begin{tikzpicture}
% \begin{axis}[
%    y tick label style={
%        /pgf/number format/.cd,
%            fixed,   % po zakomentowaniu os rzednych jest indeksowana wykladniczo
%            fixed zerofill, % 1.0 zamiast 1
%            precision=1,
%        /tikz/.cd
%    },
%    x tick label style={
%        /pgf/number format/.cd,
%            fixed,
%            fixed zerofill,
%            precision=2,
%        /tikz/.cd
%    }
% ]
% \addplot [domain=0.0:0.1] {rnd};
% \end{axis} 
% \end{tikzpicture}
% \caption{Figure caption is BELOW the figure.}
% \label{fig:3}
% \end{figure}

% \begin{figure}
% \centering
% \includegraphics[width=0.5\textwidth]{./graf/politechnika_sl_logo_bw_pion_en.pdf}
% \caption{Caption of a figure is always below the figure akakak.}
% \label{fig:3}
% \end{figure}

% some citation of the above \ref{fig:3}

% \begin{figure}
% \begin{lstlisting}
% if (_nClusters < 1)
% 	throw std::string ("unknown number of clusters");
% if (_nIterations < 1 and _epsilon < 0)
% 	throw std::string ("You should set a maximal number of iteration or minimal difference -- epsilon.");
% if (_nIterations > 0 and _epsilon > 0)
% 	throw std::string ("Both number of iterations and minimal epsilon set -- you should set either number of iterations or minimal epsilon.");
% \end{lstlisting}
% \caption{Example of pseudocode.}
% \end{figure}